\documentclass{article}
\usepackage[parfill]{parskip}
\usepackage[backend=bibtex,style=numeric-comp]{biblatex}
\usepackage[letterpaper, portrait, margin=1in]{geometry}
\bibliography{paper-webserver.bib}

\begin{document}
% "contain, at the top, the following affirmative statement. "This website is free and open to all users and there is no login requirement." Additionally, any third party software employed by the website that has more restrictive usage terms must be listed."
This website is free and open to all users and there is no login requirement. The code for this webserver, and all third party software used, are available under the open-source Apache 2.0, BSD 3-clause, or similar licenses. \par

% "include the website address; website name; and the names, affiliations, and email addresses of all authors."
The website is available at \url{https://lfz.corefacility.ca/superphy/spfy/}. Spfy's code is provided at \url{https://github.com/superphy/backend} and documentation at \url{https://superphy.readthedocs.io/en/latest/}. \par

% MAIN CONTENT
% "include a notification if this is an update from a previous publication in the Web Server issue, and in that case, include an estimate of the number of users and the number of citations."
% "For web servers, or essentially similar web servers, that have been the subject of a previous publication, including publication in journals other than NAR, there is a minimum two-year interval before re-publication in the Web Server Issue."
Our proposal covers an update to Superphy \cite{whiteside2016superphy}, an online predictive genomics platform targeting \textit{Escherichia coli}.
The update, called Spfy, uses graph data structures to store and retrieve results for computational workflows, facilitating the management and querying of tens of thousands of whole-genome \textit{E. coli} sequences, and efficient downstream processing.
% I'm unsure if we should add more about the subtyping options. For example, see:
% https://github.com/superphy/paper_platform/commit/c017b1e022d310e16a1433af9d58a73e9550a401
Current comparative computational workflows chain different analysis software, but lack storage and retrieval methods for generated results.
% "IF THE WEBSITE IMPLEMENTS A META-SERVER OR COMPUTATIONAL WORKFLOW, the summary MUST describe 1) significant added value beyond the simple chaining together of existing third party software or the calculation of a consensus prediction from third party predictors and classifiers; and at least one of the following: 2) how user time for data gathering and multi-step analysis is significantly reduced, or 3) how the website offers significantly enhanced display of the data and results."
By making the storage and retrieval of results part of the platform, with data effectively linked to the organisms of interest through a standardized ontology, we can mitigate the recomputing of analyses.
Within Spfy, the output from all analyses is stored, and linked together in the context of a genome graph. This graph also stores metadata for each genome, facilitating inquiries ranging from population genomics to epidemiological investigations.
Integrated data storage will be necessary as publicly available whole genome sequencing data for bacterial pathogens currently numbers in the tens of thousands, with hundreds of thousands set to be available within the next few years. \par

% STATISTICS{}
% "provide descriptions of the input data, the output, and the processing method; complete citations for previous publications of the method or the web server; and two to four keywords. Additionally, authors must indicate how long the server has been running, the number of inputs analyzed during testing, and an estimate of the number of individuals outside of the authors' group who have been involved in the testing."
Spfy was tested with 4,622 public \textit{E. coli} assembled genomes from Enterobase, storing every sequence and results for all included analysis modules.
Spfy provides real-time subtyping, and the results are immediately displayed to the user following their completion.
Subtyping options include O-antigen, H-antigen, Shiga-toxin 1, Shiga-toxin 2, and Intimin typing. Reference-lab tests include virulence factor and anti-microbial resistance annotation. All genomes are analyzed within the pan-genome framework of \textit{E. coli}, and results from all analyses are automatically associated with the source genome.
The resulting database had 1,333 nodes and 683,666 leaves, with 374,836,872 object properties. \par

% COMPARED TO EXISTING PLATFORMS
% This aims to be more of an implementation paragraph.
Existing scientific workflow technologies such as Galaxy \cite{goecks2010galaxy}, and pipelines such as the Bacterium Analysis Pipeline (BAP) \cite{thomsen2016bacterial} and the Integrated Rapid Infectious Disease Analysis (IRIDA) platform \url{http://www.irida.ca/} help automate the use of WGS data for public-health surveillance.
% data integration
Like IRIDA and BAP, Spfy automates workflows for users, and like Galaxy, Spfy uses task queues to distribute selected analysis. File uploads begin through the ReactJS-based website, where user-defined analyses options are selected. To these concepts we add Docker containerization for task queue workers, allowing analysis software to safely run in parallel.
% re: Matt "2. When comparing to IRIDA, BAP etc., can mention some differences with Superphy, e.g. the storage of interim result data that allows downstream integrated analysis"
For result storage, existing workflow technologies use relational tables \cite{goecks2010galaxy}, or store resulting files to disk \cite{thomsen2016bacterial}.
Because output from these programs is user-specific or transitory, results from identical comparisons are often recomputed. Additionally, output from different analyses are structured using distinct terminology and formats, which must be converted before they can be compared. Without a unified structure, these conversions quickly become impractical for broad usage. Graph-based storage of all results solves these problems.
% as before
To avoid proliferating ontologies, and to allow Spfy to integrate with existing ones, annotations from the GenEpiO \cite{griffiths2017context}, FALDO \cite{bolleman2016faldo}, and TypOn \cite{vaz2014typon} ontologies are used to describe biological data.
The entire platform is packaged using Docker-Compose, and can be recreated with a simple command. \par

% up-time
The Spfy update has been up since May 2017 and Superphy has been up since early 2016. The server accepts assembled \textit{E. coli} genomes with the \textit{.fasta} or \textit{.fna} extensions. Submissions are checked against a reference set of \textit{E. coli} gene sequences before running analyses. Outputs are displayed on the website in tables and can be downloaded as \textit{.csv} files. 
\par

% "and two to four keywords."
Keywords: Bioinformatics, Graph database, Real-time analysis
\end{document}
