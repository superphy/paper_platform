\documentclass[a4,center,fleqn]{NAR}

% Enter dates of publication
\copyrightyear{2008}
\pubdate{31 July 2009}
\pubyear{2009}
\jvolume{37}
\jissue{12}

\bibliography{paper-webserver.bib}

%\articlesubtype{This is the article type (optional)}

\begin{document}

\title{Spfy: distributed predictive genomics of E.coli with graph based result linkage}

\author{%
Corresponding Author\,$^{1,*}$,
First Co-Author\,$^{2}$
and Second Co-Author\,$^2$%
\footnote{To whom correspondence should be addressed.
Tel: +44 000 0000000; Fax: +44 000 0000000; Email: xxx@yyyy.ac.zz}}

\address{%
$^{1}$Affiliation of Corresponding Author
and
$^{2}$Affiliation of Both Co-Authors}
% Affiliation must include:
% Department name, institution name, full road and district address,
% state, Zip or postal code, country

\history{%
Received January 1, 2009;
Revised February 1, 2009;
Accepted March 1, 2009}

\maketitle

\begin{abstract}

\end{abstract}


\section{Introduction}
% new outline
% para
% 1. brief: WGS is standard
% 2. Big Problem: but tools are for individual analysis
% 3. 100,000 genomes, (lookup number for Enterobase, GenBank), how do we run analysis on it
% 4. previous methods (Galaxy/IRIDA: real-time, but no storage of results, other website examples - Denmark?), what problems they addressed
% 5. Problems that remain / lack of result storage means: recomputation, lost data, can't reference old analyses, not suited for "big data"
% 6. Solutions in general: store and increment, huga "big data" analyses, parallelization /queues
% para
% 1. Our specific problem, previous work
% 2. Why solving it is important (for Public health / research)
% 3. how we solved it
% para
% 4. benefits: rapid analyses in real-time -> huge comparisons, replace reference labs -> time & money saved, future work -> expand analyses, more genomes, more species
% 5. analyses modules -> conda -> IRIDA/Galaxy
% 6. short snippit on website link & github link

% 1. brief: WGS is standard
Whole genome sequencing (WGS) resolves the entire genetic content of an organism. WGS data can increase the resolution and sensitivity of bacterial surveillance \cite{ronholm2016navigating,lytsy2017time}, identification of potential disease mechanisms \cite{wang2014whole,yuen2015whole}, and clinical diagnoses \cite{willig2015whole,dewey2014clinical}.
% 2. Big Problem: but tools are for individual analysis
Targeted software, such as the Resistance Gene Identifier (RGI) \cite{mcarthur2013comprehensive} for antimicrobial resistance (AMR) gene prediction, Prokka for bacterial genome annotation \cite{doi:10.1093/bioinformatics/btu153}, and integrated platforms, such as the Bacterium Analysis Pipeline (BAP) \cite{thomsen2016bacterial} and the Integrated Rapid Infectious Disease Analysis (IRIDA) project \url{http://www.irida.ca/}, all leverage WGS data.
% 3. 100,000 genomes, (lookup number for Enterobase, GenBank), how do we run analysis on it
WGS generates data at a break-neck speed; for \textit{Escherichia coli} alone, the public genome databases EnteroBase \url{https://enterobase.warwick.ac.uk/} and GenBank \cite{doi:10.1093/nar/gks1195} have, respectively, 60,206 and 2,779,008 sequences uploaded.
To effectively exploit WGS "big-data" and maintain the rapid response time required by public health applications, one approach is to make results from WGS analysis integrated and progressive.
Typical bioinformatics software, such as RGI and Prokka, take single files as input, and integrated platforms, such as BAP and IRIDA, build workflows linking different analyses modules.
BAP and IRIDA begin to solve big-data challenges by offering a hosted solution which computes results in real-time, and distributes analyses across computing resources.
While effective for self-contained workflows, many comparative analyses such as predictive genomics methods would benefit from a broad WGS reference base.
To use vast amount of WGS data and maintain real-time results for comparative analyses, platforms would need store results in a way that avoids recomputation when adding new WGS data.
A resource framework in which WGS results are integrated and searchable; whereby the storage of results avoids recomputation, persists, and allows for iterative, on-going learning, will expand the capacity of current comparative bioinformatics analyses.
This increased capacity will further benefit surveillance, research, and clinical applications\par

% 1. Our specific problem, previous work
We have previously developed Superphy \cite{whiteside2016superphy}, an online predictive genomics platform targeting \textit{E. coli}.
Superphy integrates pre-computed results with domain-specific metadata to provide real-time analyses of epidemiology relations.
While this tool has been useful for the thousands of pre-computed genomes in its database, the current pace of genome sequencing requires real-time predictive genomic analyses of tens of thousands of genomes, and the long term storage and referencing of these results, something that the original SuperPhy platform was incapable of.
% 2. Why solving it is important (for Public health / research)
WGS offers improved resolution over traditional strain comparison methods, such as pulsed-field gel electrophoresis (PFGE) \cite{ronholm2016navigating}.
Though the cost of developing, approving, and transforming existing workflows from wet-lab to sequence prediction approaches is time consuming and expensive \cite{koser2012routine}, platforms can only perform real-time analyses and linkage to thousands of historical results by leveraging WGS.
% 3. how we solved it
In this study, we present an update to the SuperPhy platform, called Spfy.
The update rewrites result storage with backing by a graph database, takes a modular approach to tool integration, and distributes analyses over task queues, thereby allowing users to submit genomes and modules to run in parallel in real-time, and address code failures.
All results are stored as a series of linked nodes which enables the platform to build associations between results as they are generated. \par

% 4. benefits: rapid analyses in real-time -> huge comparisons, replace reference labs -> time & money saved, future work -> expand analyses, more genomes, more species
By integrating task distribution with graph storage, Spfy enables large-scale analyses, such as epidemiological associations between specific genotypes, biomarkers, host, source, and other metadata, and statistical significance testing of genome markers for user-defined groups.
Subtyping options are ..., pan-genome generation ..., group comparisons via Fisher's, ML, .... for E.coli.
By supporting multiple \textit{in-silico} subtyping options, the platform functions similar to a reference laboratory, with added support for big-data analyses.
Currently, the platform has been tested with XXX genome files and result storage for XX analyses modules.
Future work will focus on adding more analysis modules and supporting different species, which can be connected to the existing graph database without need for recalculation.
To complement existing platforms such as IRIDA, modules are self-contained and can easily be integrated into Galaxy \cite{goecks2010galaxy} based platforms.
The website and source code are available at \url{https://superphy.github.io/}. \par


% end of new intro

% **************************************************************
% Keep this command to avoid text of first page running into the
% first page footnotes
\enlargethispage{-65.1pt}
% **************************************************************

% Some general comments
% The NAR Database issue is more of a showcase then a rigorous exploration of software design choices.
% Given this focus, i would suggest the following:
% 1. Increase/highlight the discriptions of the functions and capabilities, maybe by adding a Functionality section
% 2. In the Implmentation (or Methods) section, only give a cursory description of the layout and components. Don't need to provide too much justification
% 3. Use the Results section to highlight the scope/size and speed. This can be short
% 4. In the Discussion, this is where i would expand on the justications and reasons for specific design choices. Pick 2-3 main ones and discuss those (i.e. don't need to justify our choice of documentation software). Also compare with other software in Discussion.
% 5. Add a conclusions section


\section{FUNCTIONALITY}

% Describe available functions in spfy
% para covering everything
Spfy performs reference laboratory tasks: O-antigen and H-antigen typing, Shiga-toxin (STX) typing,  virulence factor (VF) and antimicrobial resistance (AMR) gene determination, and related strain determination.
Spfy also performs bioinformatics analyses: pangenome generation, statistical significance testing of genome markers for user-defined groups, and AMR predictions using support vector machines (SVMs).

% para covering ectyper & RGI
To further characterize strains ... serotyping, which detects the presence of specific cell surface antigens.
O-antigen and H-antigen typing, and VF determination use our ECTyper \url{https://github.com/phac-nml/ecoli_serotyping} software.
ECTYper compares the user-submitted genomes against a curated set of sequences, an approach originally implemented in the VirulenceFinder \cite{joensen2014real} tool.
The Comprehensive Antibiotic Resistance Database (CARD) \cite{mcarthur2013comprehensive} maintains a set of known AMR genes and develops the Resistance Gene Identifier (RGI) program.
Both ECTyper and RGI are based off of BLAST \cite{pmid2231712}.

% para on phylotyper
Stx typing looks at ...
Phylotyper constructs phylogenetic trees and uses ancestral reconstruction to determine Stx-types ... others.
The software can also be used to generate trees and determine closely related strains.

% para on pangenome
Bacterial species, pan-genome ...Panseq \cite{laing2010pan}, Roary \cite{page2015roary}, chewBBACA \url{https://github.com/mickaelsilva/chewBBACA}, and Graphtyper \cite{Eggertsson148403} can compute these pan-genomic regions, and are freely available.
Spfy integrates panseq ...

% storage of results
The results from ECTyper, RGI, Phylotyper, and Panseq are converted into a graph structure and linked the the genomes they originated from.

% use of stored results
By retrieving results from the database, Spfy provides group comparisons using Fishers, and can build SVMs to predict AMR based off of the presence of particular pangenome regions.

\section{IMPLEMENTATION}
% para: the semantic web
Spfy is built around semantic web technologies which focuses on describing the relaions between different datum \cite{berners2001semantic}.
In biological data, a semantic web focus describes individual data points by the type, for example as a genome, contiguous DNA sequence, or gene, and then link related data together in a queryable graph.
Semantic web technlogies allow new, not previously described data to be seemlessly incorporated to the existing graph, and has been proposed as a common standard for the open sharing of data \cite{horrocks2005semantic}.

\subsection{Data Storage}
% para
% 0. Goals: big-data, everything linked, easy addition of new links
% 1. spfy is built around graph technologies
% 1. how we structure our graph
% 2. ontoogies used
% 3. inferencing
% 3 1/2. SPARQL queries

Graph databases focus on describing the relationships between different data, and is one of the emerging \cite{de2015trends} database types used for biological data.
Spfy 

Serotyping, VF, AMR predictions are computed within minutes, and results are efficiently stored within the graph.

\subsection{Web design}
% para
% 1. goals: intuitive/familar, ease of use 
% 2. design specs
% 3. Google Material design
% 1. implementation: reactjs, react-md, ES6, JSX
% 4. separation from Flask layer

We designed Spfy's front-end following the Material Design standards \url{https://material.io/}, released by Google.
The user interface is implemented with the React JavaScript library \url{https://facebook.github.io/react/}, by Facebook, as a single-page application to allow efficient data-flow without reloading the website.

\subsection{Real-time analysis pipelines}
% para
% 1. goals: real-time, support for pipelines (linked modules)
% 2. how pipelines have been handled in the past: Galaxy, other examples
% 3. Python, RQ
% para
% 1. how we implemented RQ
%	related: packaging of modules in conda

Spfy enables processing of thousands of genome sequences using multiple modules.
By using task queue workers, enabled by the Python-based Redis Queue library \url{https://github.com/nvie/rq}, performance can quickly scale to available infrastructure.
As a bioinformatics tool, we ensured reliability of the platform by integrating the open-source Sentry toolkit \url{https://github.com/getsentry/sentry} for real-time exception tracking.
Containers are orchestrated through Docker-Compose allowing the entire platform: webservers, databases, and task workers, to be easily replicated by other researchers.


% para
% 1. goals: scale analyses to "big-data", error handling
% 3. how we handle parallelization with RQ
% para
% 1. how many tasks have we tested this with
% 2. error handling: rq-dashboard, sentry
% 3. why options like sentry are better than traditional logging: scales well to tons (big-data levels) of tasks, groups the same errors together, reporting via email

% para
% 1. goals: why compartmentilizations
% 1. how we implemented docker
% 3. how this lets us replicate worker containers and link everything together

Everything is compartmentalized within separate Docker containers where the front-end is networked, through Docker-Compose, to the back-end webserver.

Docker integration ensures that software dependencies, which typically must be manually installed \cite{doi:10.1093/bioinformatics/btu153,laing2010pan,inouye2014srst2,naccache2014cloud}, are handled automatically.


\subsection{Continuous integration and testing}
% Keep this short
% para
% 1. goals: why CI, testing is important
% 2. how we've implemented it, integration with github 

We developed Spfy using continuous integration (CI) techniques: all changes and update to the codebase is automatically ran against a series of tests ensuring functionality and backwards compatibility.
We chose TravisCI \url{travisci.io} which offers free services for open-source projects.
The individual tests use PyTest.


\section{RESULTS}

% e.g.
% database statistics
We tested spfy with 55,353 samples genomes (267GB), storing both the entire genomes and results for all included analyses modules.
The resulting database had XYZ nodes and XYZ edges, with XYZ object properties.
This worked our to XYZ GB of data stored.

% analysis run-time / throughput with different levels of parallelization


\section{DISCUSSION}

% drawbacks - big data
Many previous bioinformatics software programs have been developed \textit{ad hoc}, without the use of software engineering principles \cite{de2015trends}.
Such tools were often script-based, with custom data formats, and only suitable for small collections of data \cite{de2015trends}.
While this was acceptable for smaller analyses, bioinformatic pipelines utilizing WGS data are larger and involve linked dependencies, which require the application of systems engineering principles \cite{schatz2015biological}.
Additionally, many subsets of biology now require the analyses of big-data, where the ability to perform computations in real-time, store data in flexible databases, and utilize a common application programming interface (API) linking resources are required \cite{swaminathan2016review}.

One of the key goals in developing Spfy is to maintain instantaneity: modern websites have accustomed users to instant results.
We attempt to use innovations in web development and bring a similar experience to Spfy as a predictive genomics platform for \textit{E. coli}.
% Mention where users / developers can find documentation
Spfy's main documentation and codebase are provided at \url{https://github.com/superphy/backend} and a developer guide is provided at \url{https://superphy.readthedocs.io/en/latest/}.

\subsection{Impact on Public Health Efforts}

% para
% focus on application
The isolation and characterization of bacterial pathogens are critical for Public Health laboratories to rapidly respond to outbreaks, and to effectively monitor known and emerging pathogens through surveillance programs.
Until recently, Public-health agencies relied on laboratory tests such as XYZ to characterize bacterial isolates in outbreak and surveillance settings.
The previous gold-standard in determining strain relatedness was pulsed-field gel electrophoresis (PFGE) {ronholm2016navigating}, which uses rare-cutting restriction enzymes to produce a unique banding pattern for each strain.
However, in \textit{Enterococcus faecium}, PFGE has been shown to misclassify 9 of 132 isolates, when compared to whole-genome sequencing (WGS) based discrimination \cite{pinholt2015multiple}.
In \textit{Klebsiella pneumoniae} \cite{marsh2015genomic}, \textit{Yersinia enterocolitica} \cite{gilpin2014limitations}, and \textit{Staphylococcus aureus} \cite{doi:10.1093/ofid/ofu096}, WGS was used to discriminate isolates after initial clustering by PFGE resulted in indistinguishable samples.
Examination of PFGE bands are also subjective, difficult to share \cite{lytsy2017time}, and collative platforms such as PulseNet reported \cite{gilpin2014limitations} that even after collecting 72\% of \textit{Campylobacter jejuni} in a given year in Minnesota (673 cases), 87\% of isolates could not be linked by PFGE pattern.
Antimicrobial resistance testing, virulence factor testing ...
However, current efforts are focused on predictive genomics, where the relevant phenotypic information can be determined through examination of the whole-genome sequence. 
, and as such can be used to evaluate the spread of outbreaks with better resolution and context than traditional methods \cite{ronholm2016navigating}.

% application: results similar to a wet-lab
Spfy uses WGS results.
After intial sequencing of new isolates, Spfy can be used in place of a traditional reference laborartory, to determine the O-type and H-type, Stx type, and all known VFs and AMR genes in real-time.
These results can be shared with other ageencies and researchers over the internet.
Futhermore, using Spfy's database of pre-processed genomes, Spfy can determine all strains a sample may be related to which is useful for ...
Cost benefits ...

\subsection{Deployment to Cloud-Computing Platforms}
% how costs can be saved because using standard web tech means you can deploy to different cloud computing services
Spfy's common infrastructure allows the platform to be deployed to multiple cloud computing services, without need to modify the code.
Docker containerization is supported by Amazon AWS, Google Cloud, ...
Neglibble impact of Docker containers on performance \cite{di2015impact}.
Costs ...

\subsection{Comparison with other bioinformatic pipeline technologies}

% namely galaxy
Other pipeline technologies such as Galaxy \cite{goecks2010galaxy}, Kepler ... \cite{de2015trends}.
important to have rapid access available for retrieval while maintaining long-term data storage \cite{schatz2015biological}
long term storage is needed to prevent the constant re-computation of the same data, and to provide pre-computed analyses for common predictive genomic analyses needed by the public health, and research community \cite{de2015trends}.
Integrating results from multiple genomes, and examining shared connections, is an area that currently requires more research \cite{fricke2014bacterial},

\subsection{Comparison with similar bioinformatic pipelines}

% para: \cite{naccache2014cloud}
% must install a ton of deps and even create dbs manually https://github.com/chiulab/surpi
% im not actually sure where they're getting the 'cloud-compatible' aspect of their paper from - its cloud compatible in that you deploy in the same manner as you would to bare-metal, but that's just like saying it 'runs on a computer'

% para: \cite{aanensen2016whole}
% makes some pretty pictures, but not really a pipeline

% para: \cite{joensen2014real,thomsen2016bacterial}
% we have a lot of similar directions to this, but just try to make it prettier, more friendly to use, and faster

\section{CONCLUSIONS}

\section{ACKNOWLEDGEMENTS}


\subsubsection{Conflict of interest statement.} None declared.
\newpage


\begin{thebibliography}{4}

% Format for article
\bibitem{1}
Author,A.B. and Author,C. (1992)
Article title.
\textit{Abbreviated Journal Name}, \textbf{5}, 300--330.

% Format for book
\bibitem{2}
Author,D., Author,E.F. and Author,G. (1995)
\textit{Book Title}.
Publisher Name, Publisher Address.

% Format for chapter in book
\bibitem{3}
Author,H. and Author,I. (2005)
Chapter title.
In
Editor,A. and Editor,B. (eds),
\textit{Book Title},
Publisher Name, Publisher Address,
pp.\ 60--80.

% Another article
\bibitem{4}
Author,Y. and Author,Z. (2002)
Article title.
\textit{Abbreviated Journal Name}, \textbf{53}, 500--520.

\end{thebibliography}

\end{document}
